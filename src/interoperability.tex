Workflow interoperability is neither a simple nor singular attribute. There
are at least four distinct types of interoperability that merit discussion:

\todo{Look at these bullets - need to be consistent with later chapters!}

\begin{enumerate}

\item Workflow interoperability---Sharing workflows across different science
problems. This was an original motivation in the initial days of eScience  and
reproducible computational science. 

\item Execution delegation---Delegating the execution of a workflow to a more
capable or appropriate workflow management system. Consider, for example, the
formal specification of a workflow as a directed acyclic graph and associated
data descriptions such that the specification is complete and thereby, in
principle, executable by any capable workflow management system.  Although
easy in principle and conception, this has proven to be less successful
in practice for at least two primary reasons: (i) Directed acyclic graphs are a
common, but not universal, formal specification of some workflows, and (ii)
many specific considerations and assumptions beyond those associated with a
directed acyclic graph need to be factored when executing workflows. These
assumptions and specific considerations in turn are often due to inadequate
infrastructure abstraction and separation of concerns.

\item Workflow system interoperability---Executing the same workflow(s) by
different workflow management systems. In addition to the absence of a
technical or formal basis for designing workflow management systems, the
sociology of software engineering and tooling contributed to the proliferation
of workflow management systems. In the presence of a proliferation of tools,
there was always a principled if not a practical demand for such workflow
system interoperability. However, even if initially a  more  ``principled'' form
of interoperability, it can be argued that workflow system interoperabilty is
increasingly important because of the needs and requirements of reproducible
science.

\item Interchangeable workflow system components---Exchanging or
concurrently using components that can be exchanged across
one or more systems.

\end{enumerate} 

A primary driver for seeking interoperability across workflow systems
has been the need to address larger scientific problems that can only be
solved with workflows that require multiple systems for complete execution.
Two successful examples of limited interoperability between workflow systems
are discussed in \cite{brooks_triquetrum:_2015} and
\cite{mandal_integrating_2007}. Notably, both projects leveraged flavors of
the Ptolemy framework, namely Triquetrum and Kepler, and delegated the
execution of workflows. 
