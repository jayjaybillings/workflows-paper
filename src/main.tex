%%This is a very basic article template.  %%There is just one section and two
%%subsections.  
\documentclass[10pt,conference,final]{IEEEtran}

\usepackage{cite} 
\usepackage{tabularx}
\usepackage[autostyle]{csquotes} 
\usepackage{url} 
\usepackage{todonotes}
\usepackage{float}
\usepackage{array}
\usepackage{tabu}

\begin{document}

\title{Toward Common Components and a Standard Model for Workflow Systems}


\author{\IEEEauthorblockN{Jay Jay Billings} \IEEEauthorblockA{Oak Ridge National
Laboratory\\ PO Box 2008 MS 6173\\ Oak Ridge, TN USA 37831\\ Email:
billingsjj@ornl.gov\\ Twitter: @jayjaybillings} \and \IEEEauthorblockN{Shantenu
Jha} \IEEEauthorblockA{RADICAL, ECE\\ Rutgers University, \\ Piscataway, NJ USA
08854\\ Email: shantenu.jha@rutgers.edu} }

\maketitle

\begin{abstract}

The study of ``workflow science'' has an open secret: Discussions about workflows and workflow management systems inevitably turn into discussions about the nature of those very topics! Too often, surveying a number of papers will result in different definitions from each paper and a strong inconsistency that leaves the reader with a skewed opinion. This inconsistency often centers around the different between several different types of workflows, including grid, modeling and simulation, uncertainty quantification, and purely conceptual workflows. This work explores this phenomenon by examining the different types of workflows and workflow management systems, reviewing the perspective of a large supercomputing facility, examining the common features and problems of workflow management systems, and finally presenting a proposed solution based on the concept of common building blocks. The implications of the continuing proliferation of workflow management systems and the lack of interoperability between these systems are discussed from a practical perspective.

\end{abstract}
\hfill \\
\underline{Notice of Copyright:} This manuscript has been authored by UT-
Battelle, LLC under Contract No. DEAC05-00OR22725 with the U.S. Department of
Energy. The United States Government retains and the publisher, by accepting
the article for publication, acknowledges that the United States Government
retains a nonexclusive, paid-up, irrevocable, world-wide license to publish or
reproduce the published form of this manuscript, or allow others to do so, for
United States Government purposes. The Department of Energy will provide
public access to these results of federally sponsored research in accordance
with the DOE Public Access Plan (http://energy.gov/downloads/doe-public-access-plan).


\chapter{Introduction}

The study of workflows, workflow management systems, and the broader
topics of so-called ``workflow science'' has revolutionized the way
humans automate activities in business, science, medicine and many
others fields, greatly improving efficiency in each. Our dependence on
workflow tools and technology is so great that some of the most common
questions in technical discussions are arguably ``How do we automate
that?'' and ``How are you going to handle your workflow?'' Workflow
tools have even ``invaded'' our homes by becoming so easy to use that
children can use them to clandestinely stock up on cookies, doll houses,
and Pokemon, \cite{_6-year-old_2016}\cite{williams_6-year-old_2017}!

However, there are many unresolved problems in workflow science that
will seriously limit the continued utility of these tools for
interdisciplinary research. There are many practical problems, some of
which are shared with other fields, but the largest issues are those of
a fundamental, theoretical nature that have never been investigated
across the entire field. For example, what is a ``workflow'' and what is
a ``workflow management system?'' Are workflows in modeling and
simulation the same as workflows in grid computing? Are scientific
workflows actually the same as business workflows given the right
theoretical abstraction? There are no generally accepted answers for
these questions and, indeed, every relevant reference for this work
provided a different definition of a workflow, (albeit some of the
definitions were similar).

These basic, fundamental questions have significant implications for
long-term sustainability, interoperability, and (workflow)
reproducibility. Interoperability in particular is interesting because
of the opportunity to leverage existing capabilities and workflows in
different systems without reproducing either the components or the
workflows in the second system. In practice, this is rarely possible.
Few workflow systems will work together because most systems are
developed with different or custom approaches, or, in many cases, were
never developed to be interoperable at all and there are no obvious ways
to ``just make it work.''

In this work, we propose that the solution to these problems stems from
the lack of a comprehensive ontology for workflow science. Specifically,
we

\begin{itemize}
\itemsep1pt\parskip0pt\parsep0pt
\item
  provide a brief literature review of workflows and efforts to develop
  taxonomies (not ontologies) for workflows and workflow management
  systems, and describe why a classic taxonomy may be insufficient.
\item
  present the Eclipse Integrated Computational Environment as an
  appropriate test bed for these ideas and contextualize it in the
  broader workflows tool space.
\item
  describe challenges the workflows community is currently facing that
  would be solved by the development of an ontology that facilitated
  interoperability.
\item
  enummerate the necessary tasks and challenges of such an effort.
\end{itemize}
\chapter{A review of workflows and taxonomies}

\section{Workflows}\label{workflows}

As mentioned previously, one of the most challenging aspects of studying
workflows is the way the vocabulary has been overloaded unintentionally.
It is somewhat clearer to understand them by considering the historial
perspective.

The use and study of workflows and the initial implementation of
workflow management systems developed in the business world with the
need to automate business processes. Lud\"{a}scher et al. ascribe the
origins of workflows and workflow management systems to ``office
automation'' trends in the 1970s, \cite{ludascher_scientific_2006}. Van Der
Aalst argues that ``workflows'' arose from the needs of businesses to not only
execute tasks, but ``to manage the flow of work through the
organization,'' and that managing workflows is the natural evolution
from the monolithic applications of the 1960s to applications that rely
on external functionality in the 1990s, \cite{van_der_aalst_application_1998}.
(One might argue that Van Der Aalst's depiction continues today with the
growth of the ``microservices'' architectural movement.) By 1995, in the
presence of many workflow tools, the Workflow Management Coalition had developed a ``standard'' definition of
workflows, \cite{hollingsworth_workflow_1993},

\begin{displayquote}
A Workflow is the automation of a business process, in whole or part, during
which documents, information or tasks are passed from one participant (a 
resource; human or machine) to another for action, according a set of 
procedural rules. 
\end{displayquote}

In the early 2000s, workflow systems started finding use in scientific
contexts where process automation was required for scientific uses
instead of traditional business uses. The focus of scientific workflows,
at the time, also shifted to focus primarily on data processing for
large ``grids'' of networked services, \cite{yu_taxonomy_2005}. Yu and Buyya
define a workflow as

\begin{displayquote}
... a collection of tasks that are processed on distributed resources in a
well-defined order to accomplish a specific goal.
\end{displayquote}

This definition is important because of what is missing: the human
element. For many in the scientific workflows community this has become
the standard definition of a workflow and the involvement of humans
results not in a single workflow, but two workflows spanned by a human.
Machines or instruments are absent from the definition as well, but in
practice many modern scientific workflows are launched automatically
when data ``comes off'' of instruments because they remain the primary
source of data in grid workflows, (c.f. - \cite{megino_panda:_2015}).

In addition to ``grid workflows,'' the scientific community started
exploring ``modeling and simulation'' workflows which focus not on data
flow, but on the orchestration of activities related to modeling and
simulation instead, sometimes on small local computers, but often on the
largest of the world's ``Leadership Class'' supercomputers. These
workflows typically fall into a subset of their more general cousins
that can be found in the grid or business communities, but unlike grid
workflows they tend to require human interaction in one way or another.
Some of these workflows are defined in the context of a particular way
of working, such as the Automation, Data, Environment, and Sharing
(ADES) model of Pizzi et al., \cite{pizzi_aiida:_2016}, the
``Design-to-Analysis'' model of Clay et al., \cite{clay_incorporating_2015}, or
the model of Billings et al. presented later in this work during the
discussion on the Eclipse Integrated Computational Environment. However,
many scientific workflows, while exceptionally well defined, remain hard
coded into dedicated environments developed for the sole purpose of
executing that single or at most a few related workflows.

Additional types of workflows in the scientific community include
workflows that process ensembles of calculations, \cite{montoya_apex_2016},
and workflows that are used for testing software.

\section{Taxonomies and
Classification}\label{taxonomies-and-classification}

There have been several efforts to classify, survey or develop taxonomies
for workflows. Yu and Buyya are the only source that provides what can
be truly considered a ``taxonomy,'' by showing the hierarchical
relationships between workflow concepts. Most of the other efforts
discussed below, while claiming to produce taxonomies, in fact produce
controlled vocabularies.

Human involvement is critical in some workflows which require adaptive
management, as shown by Han and Bussler, \cite{han_taxonomy_1998}. Their work
considers adaptive workflow management in the context of healthcare
workflows and argues that workflow technology in 2002 was incapable of
adapting sufficiently to meet the unexpected needs of medical personnel.
Along with unexpected needs (``changing environment''), they cite the
evolution of software systems (``technical advances'') as another
critical area that leads to required changes in workflow management
systems.

Han and Bussler discuss situations that lead to ``ad-hoc derivation'' of
workflows. ``Ad-hoc derivation'' in this case means generating extra
workflow steps or details, such as converting from an abstract to a
concrete workflow as Pegasus and other grid workflow systems do.
Specifically, Han and Bussler cite dynamic refinement, user involvement,
unpredictable events and erroneous situations as systems that require
the workflow to behave in an unplanned way, and for which
workflow managements systems should be prepared. Meta-models, open-point
(more commonly known as ``extension point'') or hybrid approaches are
proposed as solutions.

It is important to note that Han and Bussler consider only business
workflows and management systems, not scientific workflows and
management systems. In this context they also share some important
wisdom that is, arguably, of equal importance to scientific workflows:

\begin{displayquote}
Workflow systems do not exist for their own purposes. They
are a constituent component of a business system. A business system is usually domain
specific.
\end{displayquote}

This is an important consideration for scientific workflows because the
``business logic'' of scientific workflows is ``domain logic'' that is
highly specific to the scientific domain under consideration. This
necessarily leads to a diverse ecosystem of workflow systems.

Yu and Buyya developed a taxonomy for workflow management systems on
grids that sought to capture the architectural style while identifying
comparison criteria, \cite{yu_taxonomy_2005}. Their work is notable because it
largely avoids a discussion of applications and focuses purely on the
functional properties of the workflow management systems as they exist
on the grids. Yu and Buyya root their taxonomy on five core elements of
grid-based workflow management systems: workflow design, information
retrieval, workflow scheduling, fault tolerance, and data movement.
While many of the properties and taxonomic elements they describe seem
common to all systems, others would appear to be grid-specific at present, such
as ``Workflow QoS Constraints''. Their work also shows how thirteen common grid
workflow management systems, including Pegasus and Kepler, are covered by the
taxonomy. Like other authors, Yu and Buyya cite the lack of standardized
workflow syntax and language as sources of interoperability issues. Yu's and
Buyya's work is extremely detailed and a very helpful resource for
understanding grid workflows.

Scientific workflow management systems have flourished since their
inception, although not without significant overlap and duplication of
effort. The survey of scientific workflow management systems by Barker
and Hemert illustrates both the growth and problems while also providing
important observations and recommendations on the topic,
\cite{barker_scientific_2007}.

Barker and Hemert also provide key insights into the history of
workflow management systems as an important part of business
automation. The authors make an important comparison between traditional
business workflow management systems and their scientific counterparts,
citing in particular that traditional business workflow tools employ the
wrong abstraction for scientists They define workflows using the
``standard'' definition from the Workflow Management Coalition,
previously mentioned above.

The discussion points that Barker and Hemert raise are important because
of their continuing importance and relevance today, particularly the
need to enable programmability through standard languages instead of
custom, proprietary languages. (The reader is encouraged to read the
 entire paper for more details.) Sticking to
standards is also important and perhaps illustrated best by Barker's and
Hemert's statement that

\begin{displayquote}
If software development and tool support terminates on one proprietary 
framework, workflows will need to be re-implemented from scratch.
\end{displayquote}

This is an important point even for workflow tools that do not use
proprietary standards, but ``roll their own'' solutions. What can be
done to support those tools and reproduce those workflows once support
for continued development ends?

Notably, in their discussion about the Kepler workflow management
system, Barker and Hemert state that

\begin{displayquote}
Actors are re-usable independent blocks of computation, such as:
Web Services, database calls, etc.
\end{displayquote}

Jha and Turilli have proposed using independent ``building blocks'' as
an approach to scientific workflows which espouses a similar
relationship to infrastructure services, \cite{jha_building_2016}. The
similarity is notable and the exact relationship between Jha's blocks and
Actor-Oriented Programming merit further investigation. 
Jha and Turilli provide an ambiguous definition of workflows, stating
that workflows are both comprised of tasks and provide a description of
the resources and constraints for each task. The ambiguity rises from
the definition of ``tasks.'' If tasks are compute processes, then their
definition is equivalent to that of grid workflows. However, if a task
could include human interaction during execution, then they have a much
broader definition than what is normally found in the grid literature,
and their definition more closely resembles a business workflow.
Building blocks, on the other hand, are concretely defined as those
pieces of middleware that are self-sufficient, interoperable,
composable, and extensible, (with detailed definitions of each
provided). The RADICAL-Cybertools suite of software modules from Jha's
group is presented as a sample set of building blocks and two case
studies where these tools were successfully as such are presented.

Montoya et al. discuss workflows needs for the Alliance for Application
Performance at Extreme Scale (APEX), \cite{nersc_apex_2016}, and describe
three main classes of workflows: Simulation Science, Uncertainty
Quantification (UQ), and High Throughput Computing (HTC),
\cite{montoya_apex_2016}.
HTC workflows start with the collection of data from experiments that is in turn transported to large compute facilities for
processing. Many grid workflows are HTC workflows, but not all HTC
workflows are grid workflows since some HTC workflows, such as those
those presented by Montoya et al., maybe be run on large resources that
are not traditionally ``grid machines.'' Simulation science workflows,
referred to above as modeling and simulation workflows, are those
workflows that are primarily focused on modeling and simulation
activities. UQ workflows build on modeling and simulation workflows by
executing ensembles of jobs or ensembles of whole workflows to quantify
uncertainty in simulation results. Montoya et al. also provide a detailed
mapping of each workflow type to optimal hardware resources for the APEX
program.

The U.S. Department of Energy sponsored the \emph{DOE NGNS/CS Scientific
Workflows Workshop} on April 20-21st 2015. In the report, Deelman et al.
describe the requirements and research directions for scientific
workflows for the exascale environment, \cite{deelman_future_2015}. The report
describes scientific workflows primarily by three application types:
Simulations, Instruments, and Collaborations. The findings of the workshop are
comprehensive and encouraging, with recommendations for research
priorities in Application Requirements, Hardware Systems, System
Software, WMS Design and Execution, Programming and Usability,
Provenance Capture, Validation, and Workflow Science.

The definitions of a ``workflow'' and ``workflow management systems''
are thoroughly explored and put into context for the purposes of the
workshop. The authors of the report are very careful to define workflows
not just as a collection of managed processes, which is common, but in
such a way that it is clear that reproducibility, mobility and some
degree of generality are required by both the description of the
workflow and the management system. \footnote{The report appears to provide
three separate definitions for ``workflow'' on pages 6, 9 and 10.}

The brief summary of different workflow models above is a sample of the
confusion in the ``marketplace'' and illustrates the lack of a coherent
understanding of workflows.

The next section presents the workflow model, system architecture and
applications of the Eclipse Integrated Computational Environment. This
model limits its scope to high-performance computing (HPC) and to the
set of possible workflows that come creating input, executing jobs,
analyzing results, managing data, and modifying code. However, as ``limited'' as
ICE's model may be, it shows significant ability to interoperate with
other workflow engines. This and other qualities of the system are why
it is revisited later as a proposed platform for testing an
interoperability layer.


\section{Experience of a Leadership Computing Facility}\label{olcf}

\subsection{Proliferation and Common Functionality} \label{commonFunc}

Problems with the increase in the number of existing workflow management
systems have been illustrated well by reports and discussions surrounding the
future of workflow management in leadership computing facilities. The
proliferation of workflow management systems and lack of a consistent
definition of a workflow are significant barriers to the adoption of this
technology in these facilities. The \textit{High Performance Computing Facility
Operational Assessment 2015: Oak Ridge Leadership Computing Facility (OLCF)}
report \cite{barker_scientific_2007} describes the problem such
facilities face:  \begin{displayquote} These discussions concluded with the
observation that the current proliferation of workflow systems in response to
perceived domain-specific needs of scientific workflows makes it difficult to
choose a site-wide operational workflow manager, particularly for
leadership-class machines. However, there are opportunities where facilities
can centralize workflow technology offerings to reduce anticipated
fragmentation. This is especially true if a facility attempts to develop,
deploy, and operate each and every workflow solution requested by the user
community. Through these evaluations, the OLCF seeks to identify interesting
intersections that are of the most value to OLCF stakeholders.
\end{displayquote}  OLCF's strategy is notable because it makes a
very practical observation that the problem of proliferation can be solved by
consolidation of common functionality. This is typical of an operational
perspective where deployment of capability is more important than in-depth
investigation and research into how that capability functions.
% \section{Challenges And Problems With Workflow Models and Management Systems}\label{commonFunc}

\section{Challenges of Workflow Management Systems}\label{commonFunc}

The review of different workflow models and management systems in
\S\ref{workflows} illustrates the diversity of solutions, the lack of a coherent
understanding of workflows per se, and the absence of a coordinated search for
higher-level concepts in spite of very good past efforts. That is, there is no
standard model that describes what a workflow is, the common elements of
workflow management systems, or the description of how the pieces of such a
system interact to execute a workflow. Furthermore, there are few examples of
interoperability among existing systems in spite of significant community
pressure and calls for cross-system workflow execution. Poor or nonexistent
interoperability is almost certainly a consequence of the ``Wild West'' state of
the field.

The state of the field does not mean that there is little or no common
functionality between workflow management systems in different domains. Many
sources in the literature, including several cited previously, indicate
that the contrary is in fact true: There is significant duplication and
commonality in this space. The overlap in these technologies is rarely discussed
on its own merits, but instead it is commonly used to create large tables
comparing different systems, as in
\cite{ferreira_da_silva_characterization_nodate}. This creates a scenario where
more effort is spent discussing \textit{how} something is accomplished
versus the arguably more important question of \textit{what} must be
accomplished. 

Expanding on the concept of what must be accomplished, some primary application
(workflow) needs include (i) lowering the development burden (increasing
productivity); (ii) extensibility; (iii) transporting an application workflow
to another resource, platform, or workflow system; and (iv) providing a
conceptual framework or basis to decide which tools are suitable or optimal for
a given workflow. Similarly, beyond having clarity on the functional and
performance capabilities of a workflow system, the primary needs of users and
developers of workflow systems include (i) lowering the need to develop
components, (ii) determining which components to use and reuse, (iii) minimal
perturbation and refactoring when extending or generalizing the functionality
or use cases supported by a workflow system, (iv) and providing constant
performance across different use-case scenarios and scales.

It is worth noting that workflow systems are rarely developed to extract
(enhance) performance. They are more about coordinating different
functionality without loss of performance. High-performance and scalability is
not often a first-order concern of general workflow systems; it could however,
be a first-order concern of specialized workflow systems or specific
components (e.g, a pilot-system that is responsible for scalable and efficient
task launching and management).

A healthy balance of \textit{what} versus \textit{how} is
important, but the discussion of how particular problems are
solved in workflow science has overtaken the discussion of what must be
accomplished, creating two severe problems: 
\begin{itemize} 
  \item A ``proliferation'' of tools that largely solve the same problem in the
  same way, but with separate, competing implementations primarily delineated
  along domain, as opposed to technological, boundaries.  
  \item A general lack of interoperability and, therefore, inability to address
  larger scientific problems using hybrid combined workflows, multifacility
  workflow campaigns, or heterogeneous hardware without significant
  reimplementation.
\end{itemize}

These two problems are closely related: Tooling proliferation might not be a
problem, given sufficient resources, in the absence of calls for
interoperability between systems, and interoperability might not be an issue if
there were not so many existing systems. However, some of the most important
aspects of these problems remain separable and should be examined as such.






\section{The Solution: Common Building Blocks}\label{buildings-blocks}

The two problems detailed above are side effects of the relentless march of
progress. The traditional approach for building workflow systems has been to
build as much of the required capability as possible into the system itself,
relying very little on external services or even third party code to address
pressing issues in one or more domains. However, history has shown that
important high-level functionality slowly moves down the software stack and
into kernels, kernel services, and system libraries. Is it better at that
point to use an existing system that requires significant time and resources
to learn, or to develop yet another workflow management system with common
tools, implementing only the gaps instead?

The answer to this question is complicated by the fact that workflows
themselves have evolved. First, contemporary workflows are often the
representation of methodological advances and may be more pervasive,  short-
lived, and wide-ranging than traditional workflows. Further, they are no longer
confined to ``big science'' projects because sophisticated workflows are needed
by many types of scientific projects, which leads to diverse design features and thus
makes it unlikely that one model will be universally applicable.  The ability to prototype, test
and experiment with workflows at scale suggests a need for interfaces and
middleware services that enable the rapid development of resources. The
challenge is to provide these capabilities along with considerations of
usability and extensibility.

Jha and Turilli discuss this trend as it relates to workflows from a 
cyber-infrastructure perspective and to existing large-scale scientific workflow
efforts \cite{jha_building_2016}. They propose that, while historically
successful, monolithic workflow systems present many problems for users,
developers, and maintainers. Instead, they propose that a new ``Lego-style''
approach might work better where individual building blocks of capability
are assembled into the final workflow management system, subsystem, or
product.

More formally, a building block is a collection of functionality commonly
identified across existing workflow systems that behaves like a logically and
uniformly addressable service. Table \ref{blocks} lists six common types of
functionality that are readily observed in workflow management systems. There
are certainly additional types of functionality that are common, but for
pedagogical reasons we limit the list to the most obvious choices in a quick review of the literature previously cited.

\begin{table*}[h]
\begin{tabu} to \textwidth {|X[l]|X[l]|} \hline
\textbf{Functionality} & \textbf{Description} \tabularnewline\hline Data and
metadata management & Management of data, metadata, and general file input and
output activities whether for internal tracking or external user
consumption. \\ \hline 
Workflow execution engine & The primary
actor that manages the execution of the activities as provided by the workflow
description. \tabularnewline\hline 
Resource management and acquisition &
Acquisition and management of resources, whether computing or instrumentation,
required for the successful execution of the workflow. \tabularnewline\hline
Task management & Primary subsystem for managing individual activities, tasks or
``subworkflow'' using resources provided by the task management system. This
system is sometimes, but neither often nor exclusively, part of the workflow
execution engine. \tabularnewline\hline 
Provenance engine & System for tracking
execution history, sources, and destinations of ingested and generated artifacts,
execution metadata including status, general logging, and provenance-based
inference tools. \tabularnewline\hline 
Application programming interface (API) &
A non-functional element of most workflow management systems that is critical to
successful deployment and maintenance of the full system as well as use
as a tool for creating and executing workflows. \tabularnewline\hline
\end{tabu} 
\caption{Functionality commonly identified in workflow management
systems.} 
\label{blocks}
\end{table*}

Each of the types of functionality listed in Table
\ref{blocks} could be developed, presumably through one or more community
efforts, as a building block (even the API through some programming
trickery!). Other things like programming interfaces to queuing systems,
programmable pilot systems for scheduling jobs, workload balancers, and
ensemble execution tools, among others, could be provided as well to create a
rich ecosystem of reusable and interchangeable parts.

Reusable building blocks would greatly improve both interoperability and
sustainability because they would standardize, to some degree, the programming
interfaces and back-ends used by workflow management systems. To the extent that
projects are willing to use common building blocks, proliferation would be fully
decoupled from interoperability. Leadership computing facilities would not need
to support every workflow management system, just a set of common building
blocks. This is similar to how they support third-party libraries for software
development: they do not support every code used on these machines, but
they support a set of common libraries that the codes can use. 

There is an important practical question here: Does this mean abandoning
existing workflow management systems or redeveloping existing workflows? No, and
in fact it may be quite practical to develop building blocks based on components
of the most sophisticated workflow management systems already in existence.
Furthermore, because building blocks would naturally enable interoperability, it
is quite conceivable that a workflow that only executes on one system now may
execute on many systems in the future with little or no modification. 

A second question is whether or not building blocks represent a significantly new
type of modularity versus a traditional software stack or framework. Building blocks
arguably sit above these entities and have distinct conceptual and functional roles.
A software stack is the full set of software, including all dependencies, for a given
application or software product and a framework is the set of common functionality
(APIs, not libraries) around which the product is built. On the other hand, a
building block may be implemented using a framework and will have some software stack,
but it will also offer a complete set of functionality that can be used directly in
an application. The building block may also be offered on a different system with a
different implementation (i.e., using a different software stack and framework), but neither its functionality nor service interface would change.
\section{Discussion and The Road Ahead}\label{discussion}

This is a practice (experience) paper motivated by the widely shared
perception if not strong empirical evidence/observation  that  there is a
problem in the current practice of workflow management systems.

There is an important separation between challenges of expressing workflows
effectively (algorithm) versus a workflow system that will execute the
workflow (algorithm). This paper is about the  the practice of using workflow
systems and not expressing workflows effectively. Further, it is not a
theoretically motivated or survey paper about models of workflows or workflow
systems; plenty of such papers exist which have had limited impact on the
practice of workflow systems. 

This work describes the variety of workflows, technologies, problems, and
challenges commonly found in the workflow science space.  Self-evidently, no
single workflow management system will be able to address the next generation
of scientific challenges and practical experience dictates that a change is
necessary. We propose  that common Building Blocks are a promising and
practical solution. It proposes that a Building Blocks model solve problems of
system proliferation and interoperability by redeveloping common functionality
that exists in workflow management systems into reusable services.

An important and critical test will be to devise a validation (or negation)
test for hypothesis that a building blocks approach to workflows is in fact
more scalable, sustainable and better practice than monolithic workflow
systems. We do not harbour illusions that it will not be easy, or that it is
necessarily even possible.

It is illustrative if not instructive to understand the ecosystem of the
ABDS/Cloud Model, where there are many seemingly similar components for  data-
intensive workflows. The proliferation of components suggests there is an
strong preference of functional specialization and diversity of use, as
opposed to interoperability.  Equivalentally, there is a strong binding of
components to platforms.


\section{Acknowledgements}
The authors are grateful for the assistance and support of the following people
and institutions without which this work would not have been possible.

Mr. Billings is especially grateful for the feedback provided on this content by members of his PhD committee who are not co-authors on this paper, including Jack Dongarra, John Drake, Mike Guidry, and John Turner. Mr. Billings would also like to acknowledge the thoughtful discussions with Jim Belak on the nature of workflows in the ExAM project.

This work has been supported by the US Department of Energy, Offices of Fusion
Energy Sciences and Advanced Scientific Computing Research, and by the ORNL
Undergraduate Research Participation Program, which is sponsored by ORNL and
administered jointly by ORNL and the Oak Ridge Institute for Science and
Education (ORISE). This work has also been supported by the ASCR Leadership
Computing Program with computing time at the Oak Ridge National Laboratory
Leadership Computing Facility and the Argonne National Laboratory Leadership
Computing Facility. ORNL is managed by UT-Battelle, LLC, for the US Department
of Energy under contract no. DE-AC05-00OR22725. ORISE is managed by Oak Ridge
Associated Universities for the US Department of Energy under contract no.
DE-AC05-00OR22750.


% The bibliography 
\bibliographystyle{IEEEtran}
\bibliography{IEEEabrv,bib}

\end{document}
