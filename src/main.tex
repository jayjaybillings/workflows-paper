%%This is a very basic article template.
%%There is just one section and two subsections.
\documentclass{acmart}

\usepackage{graphicx}
\usepackage{color}
\usepackage{amsmath}
\usepackage{tabularx}
\usepackage{supertabular}
\usepackage{longtable}
\usepackage{booktabs}
\usepackage[autostyle]{csquotes}
\usepackage{listings}
\usepackage{todonotes}
\lstset{
basicstyle=\small\ttfamily,
columns=flexible
}
\usepackage{url}

\title{\todo{Need new title!}A Proposal on Workflows and the Interoperability of Workflow Management Systems}
\author{Jay Jay Billings, Shantenu Jha, Matteo Turilli}
\date{\today}

\begin{document}

\begin{abstract}
\todo{This abstract is based on the one page document that we sent to the future of scientific workflows document. I like the ideas, but it is just a place holder. We should update as needed!}

Discussions on workflows and workflow management systems inevitably turn to questions about the natures of those subjects themselves. What is a workflow? What is a workflow management system? Surveying five papers will often give five different definitions of both. This has led to what the Oak Ridge Leadership Computing Facility’s 2016 report refers to as “the current proliferation of workflow systems in response to perceived domain-specific needs of scientific workflows.” Many of these systems tend to be monolithic with few extension mechanisms for those not fully versed in the existing code base, making reuse and interoperability nigh impossible. This paper examines this phenomenon and proposes that the community should 1) focus on the development of common building blocks that can be assembled into domain-specific workflow management systems, and 2) that common building blocks are the best way to address workflow management system interoperability. This examination is performed in light of the growing need for "workflow campaigns" that execute workflows across facilities, systems and workflow types.


\todo{4) Discuss proliferation - specifically cite the recent paper from Ewa et. al, then ICE, SAW, etc. What's the difference when they actually solve the same problem?}
\todo{5) Discuss building blocks.}
\todo{6) Discuss interoperability. List types from slides. Answer "Who gives a shit?" Answer "Why do building blocks help?"}

\end{abstract}

\maketitle

\section{Notice of Copyright}\label{notice-of-copyright}

This manuscript has been authored by UT-Battelle, LLC under Contract No.
DEAC05-00OR22725 with the U.S. Department of Energy. The United States
Government retains and the publisher, by accepting the article for
publication, acknowledges that the United States Government retains a
nonexclusive, paid-up, irrevocable, world-wide license to publish or
reproduce the published form of this manuscript, or allow others to do
so, for United States Government purposes. The Department of Energy will
provide public access to these results of federally sponsored research
in accordance with the DOE Public Access Plan
(http://energy.gov/downloads/doe-public-access-plan ).

\chapter{Introduction}

The study of workflows, workflow management systems, and the broader
topics of so-called ``workflow science'' has revolutionized the way
humans automate activities in business, science, medicine and many
others fields, greatly improving efficiency in each. Our dependence on
workflow tools and technology is so great that some of the most common
questions in technical discussions are arguably ``How do we automate
that?'' and ``How are you going to handle your workflow?'' Workflow
tools have even ``invaded'' our homes by becoming so easy to use that
children can use them to clandestinely stock up on cookies, doll houses,
and Pokemon, \cite{_6-year-old_2016}\cite{williams_6-year-old_2017}!

However, there are many unresolved problems in workflow science that
will seriously limit the continued utility of these tools for
interdisciplinary research. There are many practical problems, some of
which are shared with other fields, but the largest issues are those of
a fundamental, theoretical nature that have never been investigated
across the entire field. For example, what is a ``workflow'' and what is
a ``workflow management system?'' Are workflows in modeling and
simulation the same as workflows in grid computing? Are scientific
workflows actually the same as business workflows given the right
theoretical abstraction? There are no generally accepted answers for
these questions and, indeed, every relevant reference for this work
provided a different definition of a workflow, (albeit some of the
definitions were similar).

These basic, fundamental questions have significant implications for
long-term sustainability, interoperability, and (workflow)
reproducibility. Interoperability in particular is interesting because
of the opportunity to leverage existing capabilities and workflows in
different systems without reproducing either the components or the
workflows in the second system. In practice, this is rarely possible.
Few workflow systems will work together because most systems are
developed with different or custom approaches, or, in many cases, were
never developed to be interoperable at all and there are no obvious ways
to ``just make it work.''

In this work, we propose that the solution to these problems stems from
the lack of a comprehensive ontology for workflow science. Specifically,
we

\begin{itemize}
\itemsep1pt\parskip0pt\parsep0pt
\item
  provide a brief literature review of workflows and efforts to develop
  taxonomies (not ontologies) for workflows and workflow management
  systems, and describe why a classic taxonomy may be insufficient.
\item
  present the Eclipse Integrated Computational Environment as an
  appropriate test bed for these ideas and contextualize it in the
  broader workflows tool space.
\item
  describe challenges the workflows community is currently facing that
  would be solved by the development of an ontology that facilitated
  interoperability.
\item
  enummerate the necessary tasks and challenges of such an effort.
\end{itemize}
\chapter{A review of workflows and taxonomies}

\section{Workflows}\label{workflows}

As mentioned previously, one of the most challenging aspects of studying
workflows is the way the vocabulary has been overloaded unintentionally.
It is somewhat clearer to understand them by considering the historial
perspective.

The use and study of workflows and the initial implementation of
workflow management systems developed in the business world with the
need to automate business processes. Lud\"{a}scher et al. ascribe the
origins of workflows and workflow management systems to ``office
automation'' trends in the 1970s, \cite{ludascher_scientific_2006}. Van Der
Aalst argues that ``workflows'' arose from the needs of businesses to not only
execute tasks, but ``to manage the flow of work through the
organization,'' and that managing workflows is the natural evolution
from the monolithic applications of the 1960s to applications that rely
on external functionality in the 1990s, \cite{van_der_aalst_application_1998}.
(One might argue that Van Der Aalst's depiction continues today with the
growth of the ``microservices'' architectural movement.) By 1995, in the
presence of many workflow tools, the Workflow Management Coalition had developed a ``standard'' definition of
workflows, \cite{hollingsworth_workflow_1993},

\begin{displayquote}
A Workflow is the automation of a business process, in whole or part, during
which documents, information or tasks are passed from one participant (a 
resource; human or machine) to another for action, according a set of 
procedural rules. 
\end{displayquote}

In the early 2000s, workflow systems started finding use in scientific
contexts where process automation was required for scientific uses
instead of traditional business uses. The focus of scientific workflows,
at the time, also shifted to focus primarily on data processing for
large ``grids'' of networked services, \cite{yu_taxonomy_2005}. Yu and Buyya
define a workflow as

\begin{displayquote}
... a collection of tasks that are processed on distributed resources in a
well-defined order to accomplish a specific goal.
\end{displayquote}

This definition is important because of what is missing: the human
element. For many in the scientific workflows community this has become
the standard definition of a workflow and the involvement of humans
results not in a single workflow, but two workflows spanned by a human.
Machines or instruments are absent from the definition as well, but in
practice many modern scientific workflows are launched automatically
when data ``comes off'' of instruments because they remain the primary
source of data in grid workflows, (c.f. - \cite{megino_panda:_2015}).

In addition to ``grid workflows,'' the scientific community started
exploring ``modeling and simulation'' workflows which focus not on data
flow, but on the orchestration of activities related to modeling and
simulation instead, sometimes on small local computers, but often on the
largest of the world's ``Leadership Class'' supercomputers. These
workflows typically fall into a subset of their more general cousins
that can be found in the grid or business communities, but unlike grid
workflows they tend to require human interaction in one way or another.
Some of these workflows are defined in the context of a particular way
of working, such as the Automation, Data, Environment, and Sharing
(ADES) model of Pizzi et al., \cite{pizzi_aiida:_2016}, the
``Design-to-Analysis'' model of Clay et al., \cite{clay_incorporating_2015}, or
the model of Billings et al. presented later in this work during the
discussion on the Eclipse Integrated Computational Environment. However,
many scientific workflows, while exceptionally well defined, remain hard
coded into dedicated environments developed for the sole purpose of
executing that single or at most a few related workflows.

Additional types of workflows in the scientific community include
workflows that process ensembles of calculations, \cite{montoya_apex_2016},
and workflows that are used for testing software.

\section{Taxonomies and
Classification}\label{taxonomies-and-classification}

There have been several efforts to classify, survey or develop taxonomies
for workflows. Yu and Buyya are the only source that provides what can
be truly considered a ``taxonomy,'' by showing the hierarchical
relationships between workflow concepts. Most of the other efforts
discussed below, while claiming to produce taxonomies, in fact produce
controlled vocabularies.

Human involvement is critical in some workflows which require adaptive
management, as shown by Han and Bussler, \cite{han_taxonomy_1998}. Their work
considers adaptive workflow management in the context of healthcare
workflows and argues that workflow technology in 2002 was incapable of
adapting sufficiently to meet the unexpected needs of medical personnel.
Along with unexpected needs (``changing environment''), they cite the
evolution of software systems (``technical advances'') as another
critical area that leads to required changes in workflow management
systems.

Han and Bussler discuss situations that lead to ``ad-hoc derivation'' of
workflows. ``Ad-hoc derivation'' in this case means generating extra
workflow steps or details, such as converting from an abstract to a
concrete workflow as Pegasus and other grid workflow systems do.
Specifically, Han and Bussler cite dynamic refinement, user involvement,
unpredictable events and erroneous situations as systems that require
the workflow to behave in an unplanned way, and for which
workflow managements systems should be prepared. Meta-models, open-point
(more commonly known as ``extension point'') or hybrid approaches are
proposed as solutions.

It is important to note that Han and Bussler consider only business
workflows and management systems, not scientific workflows and
management systems. In this context they also share some important
wisdom that is, arguably, of equal importance to scientific workflows:

\begin{displayquote}
Workflow systems do not exist for their own purposes. They
are a constituent component of a business system. A business system is usually domain
specific.
\end{displayquote}

This is an important consideration for scientific workflows because the
``business logic'' of scientific workflows is ``domain logic'' that is
highly specific to the scientific domain under consideration. This
necessarily leads to a diverse ecosystem of workflow systems.

Yu and Buyya developed a taxonomy for workflow management systems on
grids that sought to capture the architectural style while identifying
comparison criteria, \cite{yu_taxonomy_2005}. Their work is notable because it
largely avoids a discussion of applications and focuses purely on the
functional properties of the workflow management systems as they exist
on the grids. Yu and Buyya root their taxonomy on five core elements of
grid-based workflow management systems: workflow design, information
retrieval, workflow scheduling, fault tolerance, and data movement.
While many of the properties and taxonomic elements they describe seem
common to all systems, others would appear to be grid-specific at present, such
as ``Workflow QoS Constraints''. Their work also shows how thirteen common grid
workflow management systems, including Pegasus and Kepler, are covered by the
taxonomy. Like other authors, Yu and Buyya cite the lack of standardized
workflow syntax and language as sources of interoperability issues. Yu's and
Buyya's work is extremely detailed and a very helpful resource for
understanding grid workflows.

Scientific workflow management systems have flourished since their
inception, although not without significant overlap and duplication of
effort. The survey of scientific workflow management systems by Barker
and Hemert illustrates both the growth and problems while also providing
important observations and recommendations on the topic,
\cite{barker_scientific_2007}.

Barker and Hemert also provide key insights into the history of
workflow management systems as an important part of business
automation. The authors make an important comparison between traditional
business workflow management systems and their scientific counterparts,
citing in particular that traditional business workflow tools employ the
wrong abstraction for scientists They define workflows using the
``standard'' definition from the Workflow Management Coalition,
previously mentioned above.

The discussion points that Barker and Hemert raise are important because
of their continuing importance and relevance today, particularly the
need to enable programmability through standard languages instead of
custom, proprietary languages. (The reader is encouraged to read the
 entire paper for more details.) Sticking to
standards is also important and perhaps illustrated best by Barker's and
Hemert's statement that

\begin{displayquote}
If software development and tool support terminates on one proprietary 
framework, workflows will need to be re-implemented from scratch.
\end{displayquote}

This is an important point even for workflow tools that do not use
proprietary standards, but ``roll their own'' solutions. What can be
done to support those tools and reproduce those workflows once support
for continued development ends?

Notably, in their discussion about the Kepler workflow management
system, Barker and Hemert state that

\begin{displayquote}
Actors are re-usable independent blocks of computation, such as:
Web Services, database calls, etc.
\end{displayquote}

Jha and Turilli have proposed using independent ``building blocks'' as
an approach to scientific workflows which espouses a similar
relationship to infrastructure services, \cite{jha_building_2016}. The
similarity is notable and the exact relationship between Jha's blocks and
Actor-Oriented Programming merit further investigation. 
Jha and Turilli provide an ambiguous definition of workflows, stating
that workflows are both comprised of tasks and provide a description of
the resources and constraints for each task. The ambiguity rises from
the definition of ``tasks.'' If tasks are compute processes, then their
definition is equivalent to that of grid workflows. However, if a task
could include human interaction during execution, then they have a much
broader definition than what is normally found in the grid literature,
and their definition more closely resembles a business workflow.
Building blocks, on the other hand, are concretely defined as those
pieces of middleware that are self-sufficient, interoperable,
composable, and extensible, (with detailed definitions of each
provided). The RADICAL-Cybertools suite of software modules from Jha's
group is presented as a sample set of building blocks and two case
studies where these tools were successfully as such are presented.

Montoya et al. discuss workflows needs for the Alliance for Application
Performance at Extreme Scale (APEX), \cite{nersc_apex_2016}, and describe
three main classes of workflows: Simulation Science, Uncertainty
Quantification (UQ), and High Throughput Computing (HTC),
\cite{montoya_apex_2016}.
HTC workflows start with the collection of data from experiments that is in turn transported to large compute facilities for
processing. Many grid workflows are HTC workflows, but not all HTC
workflows are grid workflows since some HTC workflows, such as those
those presented by Montoya et al., maybe be run on large resources that
are not traditionally ``grid machines.'' Simulation science workflows,
referred to above as modeling and simulation workflows, are those
workflows that are primarily focused on modeling and simulation
activities. UQ workflows build on modeling and simulation workflows by
executing ensembles of jobs or ensembles of whole workflows to quantify
uncertainty in simulation results. Montoya et al. also provide a detailed
mapping of each workflow type to optimal hardware resources for the APEX
program.

The U.S. Department of Energy sponsored the \emph{DOE NGNS/CS Scientific
Workflows Workshop} on April 20-21st 2015. In the report, Deelman et al.
describe the requirements and research directions for scientific
workflows for the exascale environment, \cite{deelman_future_2015}. The report
describes scientific workflows primarily by three application types:
Simulations, Instruments, and Collaborations. The findings of the workshop are
comprehensive and encouraging, with recommendations for research
priorities in Application Requirements, Hardware Systems, System
Software, WMS Design and Execution, Programming and Usability,
Provenance Capture, Validation, and Workflow Science.

The definitions of a ``workflow'' and ``workflow management systems''
are thoroughly explored and put into context for the purposes of the
workshop. The authors of the report are very careful to define workflows
not just as a collection of managed processes, which is common, but in
such a way that it is clear that reproducibility, mobility and some
degree of generality are required by both the description of the
workflow and the management system. \footnote{The report appears to provide
three separate definitions for ``workflow'' on pages 6, 9 and 10.}

The brief summary of different workflow models above is a sample of the
confusion in the ``marketplace'' and illustrates the lack of a coherent
understanding of workflows.

The next section presents the workflow model, system architecture and
applications of the Eclipse Integrated Computational Environment. This
model limits its scope to high-performance computing (HPC) and to the
set of possible workflows that come creating input, executing jobs,
analyzing results, managing data, and modifying code. However, as ``limited'' as
ICE's model may be, it shows significant ability to interoperate with
other workflow engines. This and other qualities of the system are why
it is revisited later as a proposed platform for testing an
interoperability layer.



\todo{1) Discuss types of workflows.}
\todo{2) Discuss types of workflow management systems.}
\todo{3) Discuss "thin red line" slide on difference between a point-solution and a real WMS.}

\chapter{Problems}

The previous review of workflows and workflow management systems as well as the
discussion of Eclipse ICE illustrated the diverse nature of scientific
workflows. There are many problems that arise from this diversity the community
would stand to benefit if they could be addressed. However, it should be clear
from the previous sections that there are two significant problems facing the workflow science community:
\begin{itemize}
  \item Everyone has their own workflow tools, and
no two tool chains are the same.
  \item There is very little interoperability in the workflow tool space.
\end{itemize}

The following sections elaborate on community drivers for providing
interoperability between workflow management systems.

\section{A Buildings Block Approach}\label{a-buildings-block-approach}

For many reasons, the traditional approach for building workflow systems
has been to build as much of the required capability as possible into
the system itself, relying very little on external services or even
third party code. This does not track will with the course of history
since high-level functionality tends to slowly creep down the
software stack and into kernels, kernel services, and system libraries. Jha and
Turilli discuss this trend as it relates to workflows from a cyber
infrastructure perspective and to existing large scale scientific
workflow efforts, \cite{jha_building_2016}. They propose that, while historically
successful, monolithic workflow systems present many problems for users,
developers, and maintainers. Instead, they propose that a new ``Lego
style'' approach might work better where individual ``building blocks''
of capability are assembled into the final workflow product. These
building blocks would include things like programming interfaces to
queuing systems, programmable pilot systems for scheduling jobs,
workload balancers, and ensemble execution tools, among others.

This approach would greatly improve both interoperability and sustainability
because it would standardize the programming interfaces and backends used by
workflow management systems. Jha et al. are developing a white paper to address
this further, \cite{jha_towards_2016}.

\section{SOS20 Discussion on
Workflows}\label{sos20-discussion-on-workflows}

Session IV of the Twentieth Anniversary Meeting of the SOS Workshop
(SOS20) focused on workflow and workflow management
system development activities of the three participating institutions: Sandia
National Laboratory, Oak Ridge National Laboratory, and the Swiss National Supercomputing Centre, \cite{pack_sos20_2016}. Multiple
presenters illustrated the challenges facing the workflow science community and
organized a separate, spontaneous ``Birds of a Feather'' (BOF) session at
the end of the event. The BOF attendees widely agreed after a thorough
discussion that no one single workflow management system could satisfy
all the needs of those present, but that the community as a whole would
be served best by seeking to enable interoperability where possible. This
discussion led to a wider engagement of the community by BOF attendees
who decided that it would be best to engage in an open public forum so that
other groups could join the effort, \cite{billings_[science-iwg]_2016}.

\section{High-Performance Computing
Drivers}\label{high-performance-computing-drivers}

The large number of workflow management systems and lack of
interoperability is a significant barrier to their adoption in
high-performance computing facilities such as the U.S. Department of Energy's
Leadership Computing Facilities. The high performance computing Facility
Operational Assessment 2015: Oak Ridge Leadership Computing Facility
report, \cite{barker_scientific_2007}, illustrates the problem that such facilities
face,

\begin{displayquote}
These discussions concluded with the observation that the current proliferation
of workflow systems in response to perceived domain-specific needs of 
scientific workflows makes it difficult to choose a site-wide operational
workflow manager, particularly for the leadership-class machines. However,
there are opportunities where facilities can centralize workflow technology 
offerings to reduce anticipated fragmentation. This is especially true if a 
facility attempts to develop, deploy, and operate each and every workflow 
solution requested by the user community. Through these evaluations, the OLCF
seeks to identify interesting intersections that are of the most value to OLCF
stakeholders.
\end{displayquote}

The Oak Ridge Leadership Computing Facility (OLCF) is not alone in this
as other Leadership Computing Facilities face the same problem. Few
of these facilities have discussed the possibility they may end up
supporting different workflows systems entirely such that workflows at
one facility can not be run at another without significant work to
install an additional workflow management system!

Developing interoperability layers based on common building blocks with
standard service interfaces would address both of these problems by
allowing facilities to run various workflows on the system that worked
best for their hardware and software configuration.

\section{Energy Science Drivers}\label{energy-science-drivers}

The Future of Scientific Workflows report by Deelman et al. illustrates
the idea of the ``large-scale science campaign,'' \cite{deelman_future_2015}.
Such a campaign integrates multiple workflows to perform data
acquisition from experimental equipment, modeling and analysis with
supercomputers, and data analysis with either grids computing or
supercomputers. This is, in essence, the same problem that
PanDA/BigPanDA solves for the Large Hadron Collider. So, while it is not
a new problem, it is increasingly common. One solution to this problem
is to support all the possible workflow management systems, but that
becomes quickly unsustainable, as the OLCF learned through experience.
Enabling interoperability is a much more sustainable solution.

An interesting case study is the ICEMAN project at Oak Ridge National
Laboratory, on which the author is a co-principal investigator. ICEMAN
seeks to use a combination of instruments from the Spallation Neutron
Source, OLCF compute resources such as the Titan supercomputer, the
Laboratory's Compute Advanced Data Environment for Science (CADES),
Eclipse ICE, and AiiDA (c.f. - \cite{pizzi_aiida:_2016}) to automate the
analysis of quasielastic neutron scattering data and provide a
comprehensive problem solving environment for scientists.

\section{The Solution: Common Building Blocks}\label{buildings-blocks}

The rich landscape of workflow tools suggests that the  challenge is not to
provide only "raw capabilities". There exist many tools and systems that
provide the raw capabilities. The challenge is to provide these capabilities
along with considerations of usability and extensibility.

For many reasons, the traditional approach for building workflow systems has
been to build as much of the required capability as possible into the system
itself, relying very little on external services or even third party code.
This does not track well with the course of history since high-level
functionality tends to slowly creep down the software stack and into kernels,
kernel services, and system libraries.

Historically, workflow systems were developed to support “big science”
projects such as the LHC Experiments, LIGO Project and other community-wide
projects. Big Science projects were (and are still) characterized by the
availability of "workflow specialists" available to support the  development
and deployment of community workflow systems. Further, community-wide projects
were characterized by a select few important workflows, which were the primary
recipients of specialists' efforts. Once developed, a workflow system would
be used by hundreds of researchers over multiple years.

However, workflows aren’t what they used to be, for multiple reasons. First
workflows are often the representation of methodological advances and are thus
more pervasive, short-lived and wide-ranging. Further, they are no longer
confined to “big science” projects and sophisticated workflows are needed by
multiple science projects, which leads to diverse “design points” and thus
making it unlikely that “one size fits all”.  The ability to prototype, test
and experiment with workflows at scale suggests a need for interfaces and
middleware services that enable the rapid development of resources.

The proliferation of workflows and uptake by a wider range of science projects
(long-tail) implies that there is a need to support both end-users of worklows,
as well as workflow system and tool developers!

Jha and Turilli discuss this trend as it relates to workflows from a 
cyber-infrastructure perspective and to existing large-scale scientific workflow
efforts, \cite{jha_building_2016}. They propose that, while historically
successful, monolithic workflow systems present many problems for users,
developers, and maintainers. Instead, they propose that a new ``Lego style''
approach might work better where individual ``building blocks'' of capability
are assembled into the final workflow product. These building blocks would
include things like programming interfaces to queuing systems, programmable
pilot systems for scheduling jobs, workload balancers, and ensemble execution
tools, among others.

This approach would greatly improve both interoperability and sustainability
because it would standardize the programming interfaces and backends used by
workflow management systems. Jha et al. are developing a white paper to address
this further, \cite{jha_towards_2016}.


\section{Acknowledgements}
The authors are grateful for the assistance and support of the following people
and institutions without which this work would not have been possible.

Mr. Billings is especially grateful for the feedback provided on this content by members of his PhD committee who are not co-authors on this paper, including Jack Dongarra, John Drake, Mike Guidry, and John Turner. Mr. Billings would also like to acknowledge the thoughtful discussions with Jim Belak on the nature of workflows in the ExAM project.

This work has been supported by the US Department of Energy, Offices of Fusion
Energy Sciences and Advanced Scientific Computing Research, and by the ORNL
Undergraduate Research Participation Program, which is sponsored by ORNL and
administered jointly by ORNL and the Oak Ridge Institute for Science and
Education (ORISE). This work has also been supported by the ASCR Leadership
Computing Program with computing time at the Oak Ridge National Laboratory
Leadership Computing Facility and the Argonne National Laboratory Leadership
Computing Facility. ORNL is managed by UT-Battelle, LLC, for the US Department
of Energy under contract no. DE-AC05-00OR22725. ORISE is managed by Oak Ridge
Associated Universities for the US Department of Energy under contract no.
DE-AC05-00OR22750.


\bibliographystyle{ACM-Reference-Format}
\bibliography{bib}

\end{document}
