\section{Discussion and The Road Ahead}\label{discussion}

This paper is about the practice of using workflow systems in general, as
opposed to the experience of a specific workflow system. It is motivated by
the widely shared perception, if not strong empirical evidence and observation
that there is a problem in the current practice of workflow management
systems. The paper describes a variety of problems and challenges commonly
found in the workflow science space. Self-evidently, no single workflow
management system will be able to address the next generation of scientific
challenges and practical experience dictates that a change is necessary.

There is an important separation between the challenges of expressing
workflows effectively versus a workflow system that will execute the workflow.
In this paper, we do not discuss the challenges inherent in expressing
workflows effectively. Further, this work is not a theoretically motivated or
survey paper about models of workflows or workflow systems; although plenty of
such papers exist, their impact on the practice of workflow systems has been
limited.

It is illustrative if not instructive to understand the ecosystem of the
Apache BigData Software Stack/Cloud Model, where there are many seemingly
similar components for  data-intensive workflows. The proliferation of
components suggests there is an strong preference of functional specialization
and diversity of use, as opposed to interoperability.  Equivalentally, there
is a strong binding of components to platforms.

In response to the problems and experience, we propose that common components
in the form of building blocks are a promising and practical solution. We
suggest that a building blocks approach will solve problems of system
proliferation and interoperability by harnessing and developing common
functionality that exists in workflow management systems into reusable
services.

An important and critical test will be to devise a validation (or negation)
test for the hypothesis that a building blocks approach to workflows is in fact
more scalable, sustainable and better practice than monolithic workflow
systems. We do not harbor illusions that it will be easy, or that it is
necessarily even possible.



% This is a practice (experience) paper motivated by the widely shared perception if not strong empirical evidence or observation that there is a problem in the current practice of workflow management systems.
% There is an important separation between the challenges of expressing workflows effectively versus a workflow system that will execute the workflow. This paper is about the the practice of using workflow systems; it is not about expressing workflows effectively. Further, it is not a theoretically moti- vated or survey paper about models of workflows or workflow systems; although plenty of such papers exist, their impact on the practice of workflow systems has been limited.
% This work describes the variety of workflows, technologies, problems, and challenges commonly found in the workflow science space. Self-evidently, no single workflow management system will be able to address the next generation of scientific challenges and practical experience dictates that a change is necessary.
% It is illustrative if not instructive to understand the ecosys- tem of the Apache BigData Software Stack/Cloud Model, where there are many seemingly similar components for data- intensive workflows. The proliferation of components suggests there is an strong preference of functional specialization and diversity of use, as opposed to interoperability. Equivalentally, there is a strong binding of components to platforms.
% We propose that common Building Blocks are a promising and practical solution. We propose that a Building Blocks model solve problems of system proliferation and interoper- ability by redeveloping common functionality that exists in workflow management systems into reusable services.
% An important and critical test will be to devise a validation (or negation) test for the hypothesis that a building blocks approach to workflows is in fact more scalable, sustainable and better practice than monolithic workflow systems. We do not harbour illusions that it will be easy, or that it is necessarily even possible.