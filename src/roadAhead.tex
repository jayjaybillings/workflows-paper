\section{The Road Ahead}

When it comes to interoperability there are three distinct types/levels of interoperability that merit discussion:

\begin{itemize}

\item workflow interoperability (sharing workflows across different science
problems)

\item workflow system interoperability (a workflow can be executed by multiple
different workflow management systems)

\item workflow management systems can exchange components, or can be used
concurrently.

\end{itemize}

This paper is not a lament about the lack of interoperability; others have done so 
adequately if not eloquently.

% This paper/work is not about a "workflow commons" where people will share  


{\bf Are there applications that would not have happened without a workflow
system, or a specific workflow sytem. Or are workflow systems mostly
incidental in the achievement of a scientific objective.} Thus, it is
important to ask why do scientists use general purpose workflow sytems.
Definitely not to add another layer of complexity or possible degree of
failure.

\begin{itemize}

\item Manage the coordination of many components
\item Lower the burden to achieve some functionality
\item .....

\end{itemize}


