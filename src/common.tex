% \section{Challenges And Problems With Workflow Models and Management Systems}\label{commonFunc}

\section{Challenges of Workflow Management Systems}\label{commonFunc}

The review of different workflow models and management systems in
\S\ref{workflows} illustrates the diversity of solutions, the lack of a coherent
understanding of workflows per se, and the absence of a coordinated search for
higher-level concepts in spite of very good past efforts. That is, there is no
standard model that describes what a workflow is, the common elements of
workflow management systems, or the description of how the pieces of such a
system interact to execute a workflow. Furthermore, there are few examples of
interoperability among existing systems in spite of significant community
pressure and calls for cross-system workflow execution. Poor or nonexistent
interoperability is almost certainly a consequence of the ``Wild West'' state of
the field.

The state of the field does not mean that there is little or no common
functionality between workflow management systems in different domains. Many
sources in the literature, including several cited previously, indicate
that the contrary is in fact true: There is significant duplication and
commonality in this space. The overlap in these technologies is rarely discussed
on its own merits, but instead it is commonly used to create large tables
comparing different systems, as in
\cite{ferreira_da_silva_characterization_nodate}. This creates a scenario where
more effort is spent discussing \textit{how} something is accomplished
versus the arguably more important question of \textit{what} must be
accomplished. 

Expanding on the concept of what must be accomplished, some primary application
(workflow) needs include (i) lowering the development burden (increasing
productivity); (ii) extensibility; (iii) transporting an application workflow
to another resource, platform, or workflow system; and (iv) providing a
conceptual framework or basis to decide which tools are suitable or optimal for
a given workflow. Similarly, beyond having clarity on the functional and
performance capabilities of a workflow system, the primary needs of users and
developers of workflow systems include (i) lowering the need to develop
components, (ii) determining which components to use and reuse, (iii) minimal
perturbation and refactoring when extending or generalizing the functionality
or use cases supported by a workflow system, (iv) and providing constant
performance across different use-case scenarios and scales.

It is worth noting that workflow systems are rarely developed to extract
(enhance) performance. They are more about coordinating different
functionality without loss of performance. High-performance and scalability is
not often a first-order concern of general workflow systems; it could however,
be a first-order concern of specialized workflow systems or specific
components (e.g, a pilot-system that is responsible for scalable and efficient
task launching and management).

A healthy balance of \textit{what} versus \textit{how} is
important, but the discussion of how particular problems are
solved in workflow science has overtaken the discussion of what must be
accomplished, creating two severe problems: 
\begin{itemize} 
  \item A ``proliferation'' of tools that largely solve the same problem in the
  same way, but with separate, competing implementations primarily delineated
  along domain, as opposed to technological, boundaries.  
  \item A general lack of interoperability and, therefore, inability to address
  larger scientific problems using hybrid combined workflows, multifacility
  workflow campaigns, or heterogeneous hardware without significant
  reimplementation.
\end{itemize}

These two problems are closely related: Tooling proliferation might not be a
problem, given sufficient resources, in the absence of calls for
interoperability between systems, and interoperability might not be an issue if
there were not so many existing systems. However, some of the most important
aspects of these problems remain separable and should be examined as such.

Workflow interoperability is neither a simple nor singular attribute. There
are at least four distinct types of interoperability that merit discussion:

\todo{Look at these bullets - need to be consistent with later chapters!}

\begin{enumerate}

\item Workflow interoperability---Sharing workflows across different science
problems. This was an original motivation in the initial days of eScience  and
reproducible computational science. 

\item Execution delegation---Delegating the execution of a workflow to a more
capable or appropriate workflow management system. Consider, for example, the
formal specification of a workflow as a directed acyclic graph and associated
data descriptions such that the specification is complete and thereby, in
principle, executable by any capable workflow management system.  Although
easy in principle and conception, this has proven to be less successful
in practice for at least two primary reasons: (i) Directed acyclic graphs are a
common, but not universal, formal specification of some workflows, and (ii)
many specific considerations and assumptions beyond those associated with a
directed acyclic graph need to be factored when executing workflows. These
assumptions and specific considerations in turn are often due to inadequate
infrastructure abstraction and separation of concerns.

\item Workflow system interoperability---Executing the same workflow(s) by
different workflow management systems. In addition to the absence of a
technical or formal basis for designing workflow management systems, the
sociology of software engineering and tooling contributed to the proliferation
of workflow management systems. In the presence of a proliferation of tools,
there was always a principled if not a practical demand for such workflow
system interoperability. However, even if initially a  more  ``principled'' form
of interoperability, it can be argued that workflow system interoperabilty is
increasingly important because of the needs and requirements of reproducible
science.

\item Interchangeable workflow system components---Exchanging or
concurrently using components that can be exchanged across
one or more systems.

\end{enumerate} 

A primary driver for seeking interoperability across workflow systems
has been the need to address larger scientific problems that can only be
solved with workflows that require multiple systems for complete execution.
Two successful examples of limited interoperability between workflow systems
are discussed in \cite{brooks_triquetrum:_2015} and
\cite{mandal_integrating_2007}. Notably, both projects leveraged flavors of
the Ptolemy framework, namely Triquetrum and Kepler, and delegated the
execution of workflows. 






