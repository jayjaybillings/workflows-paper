\section{Experience of a Leadership Computing Facility}\label{olcf}

\subsection{Proliferation and Common Functionality} \label{commonFunc}

Problems with the increase in the number of existing workflow management
systems have been illustrated well by reports and discussions surrounding the
future of workflow management in the leadership computing facilities. The
proliferation of workflow management systems and lack of a consistent
definition of a workflow are significant barriers to the adoption of this
technology in these facilities. The \textit{High Performance Computing Facility
Operational Assessment 2015: Oak Ridge Leadership Computing Facility (OLCF)}
report \cite{barker_scientific_2007} describes the problem that such
facilities face:  \begin{displayquote} These discussions concluded with the
observation that the current proliferation of workflow systems in response to
perceived domain-specific needs of scientific workflows makes it difficult to
choose a site-wide operational workflow manager, particularly for
leadership-class machines. However, there are opportunities where facilities
can centralize workflow technology offerings to reduce anticipated
fragmentation. This is especially true if a facility attempts to develop,
deploy, and operate each and every workflow solution requested by the user
community. Through these evaluations, the OLCF seeks to identify interesting
intersections that are of the most value to OLCF stakeholders.
\end{displayquote}  OLCF's strategy is notable because it makes a
very practical observation that the problem of proliferation can be solved by
consolidation of common functionality. This is typical of an operational
perspective where deployment of capability is more important than in-depth
investigation and research into how that capability functions.