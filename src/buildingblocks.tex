\subsection{Common Building Blocks}\label{buildings-blocks}

The two problems detailed above are side effects of the relentless march of
progress. The traditional approach for building workflow systems has been to
build as much of the required capability as possible into the system itself,
relying very little on external services or even third party code to address
pressing issues in one or more domains. However, history has shown that
important high-level functionality slowly moves down the software stack and
into kernels, kernel services, and system libraries. Is it better at that
point to use an existing system that requires significant time and resources
to learn, or to develop yet another workflow management system with common
tools, implementing only the gaps instead? The community has arguably chosen
both!

Contemporary workflows are no longer confined to ``big science'' projects,
which leads to diverse design features and thus makes it unlikely that one
model will be universally applicable. The ability to prototype, test and
experiment with workflows at scale suggests a need for interfaces and
middleware services that enable the rapid development of resources. The
challenge is to provide these capabilities along with considerations of
usability and extensibility.

Jha and Turilli discuss this trend as it relates to workflows from a 
cyber-infrastructure perspective and to existing large-scale scientific
workflow efforts \cite{jha_building_2016}. They propose that, while
historically successful, monolithic workflow systems present many problems for
users, developers, and maintainers. Instead, they propose that a new
``Lego-style'' approach might work better where individual building blocks of
capability are assembled into the final workflow management system, subsystem,
or product.

More formally, a building block is a collection of functionality commonly
identified across existing workflow systems that behaves like a logically and
uniformly addressable service. Table \ref{blocks} lists six common types of
functionality that are readily observed in workflow management systems
(although others may exist as well). Reusable building blocks would greatly
improve both interoperability and sustainability because they would
standardize, to some degree, the programming interfaces and back-ends used by
workflow management systems.

\begin{table*}[h]
\begin{tabu} to \textwidth {|X[l]|X[l]|} \hline
\textbf{Functionality} & \textbf{Description} \tabularnewline\hline Data and
metadata management & Management of data, metadata, and general file input and
output activities whether for internal tracking or external user
consumption. \\ \hline 
Workflow execution engine & The primary
actor that manages the execution of the activities as provided by the workflow
description. \tabularnewline\hline 
Resource management and acquisition &
Acquisition and management of resources, whether computing or instrumentation,
required for the successful execution of the workflow. \tabularnewline\hline
Task management & Primary subsystem for managing individual activities, tasks or
``subworkflow'' using resources provided by the task management system. This
system is sometimes, but neither often nor exclusively, part of the workflow
execution engine. \tabularnewline\hline 
Provenance engine & System for tracking
execution history, sources, and destinations of ingested and generated artifacts,
execution metadata including status, general logging, and provenance-based
inference tools. \tabularnewline\hline 
Application programming interface (API) &
A non-functional element of most workflow management systems that is critical to
successful deployment and maintenance of the full system as well as use
as a tool for creating and executing workflows. \tabularnewline\hline
\end{tabu} 
\caption{Functionality commonly identified in workflow management
systems.} 
\label{blocks}
\end{table*}

In practice, it may be that numerous de facto ``common" building blocks
already exist because they may form the basis of the most sophisticated
workflow management systems already in existence. Furthermore, because building
blocks would naturally enable interoperability, it is quite conceivable that a
workflow that only executes on one system now may execute on many systems in
the future with little or no modification.

