A number of community calls have been made for interoperability. For example,
Session IV of the Twentieth Anniversary Meeting of the SOS Workshop (SOS20)
focused on workflow and workflow management system development activities of
the three participating institutions: Sandia National Laboratories, Oak Ridge
National Laboratory (ORNL), and the Swiss National Supercomputing Centre
\cite{pack_sos20_2016}. Multiple presenters illustrated the challenges facing
the workflow science community and widely agreed that no single workflow
management system could satisfy all the needs of those present. Instead,
attendees proposed that the community as a whole would be served best by
seeking to enable interoperability where possible.

Workflow interoperability is not just a conceptual attribute; it has
important practical implications. For example, DOE leadership computing
facilities, as in \S\ref{commonFunc}, are affected by the lack of
interoperabilty of all types. Consider the possibility that every facility
could end up supporting different workflows systems entirely so that workflows
at one facility cannot be run at another without significant work to install
one or more additional workflow management systems! This idea is also
illustrated well in \textit{The Future of Scientific Workflows} report through
the concept of the ``large-scale science campaign'' \cite{deelman_future_2015}.
Such a campaign integrates multiple workflows, not necessarily all in the same
workflow management system or at the same facility, to perform data
acquisition from experimental equipment, modeling and analysis with
supercomputers, and data analysis with either grid computing or
supercomputers.