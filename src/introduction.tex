\chapter{Introduction}

The study of workflows, workflow management systems, and the broader
topics of so-called ``workflow science'' has revolutionized the way
humans automate activities in business, science, medicine and many
others fields, greatly improving efficiency in each. Our dependence on
workflow tools and technology is so great that some of the most common
questions in technical discussions are arguably ``How do we automate
that?'' and ``How are you going to handle your workflow?'' Workflow
tools have even ``invaded'' our homes by becoming so easy to use that
children can use them to clandestinely stock up on cookies, doll houses,
and Pokemon, \cite{_6-year-old_2016}\cite{williams_6-year-old_2017}!

However, there are many unresolved problems in workflow science that
will seriously limit the continued utility of these tools for
interdisciplinary research. There are many practical problems, some of
which are shared with other fields, but the largest issues are those of
a fundamental, theoretical nature that have never been investigated
across the entire field. For example, what is a ``workflow'' and what is
a ``workflow management system?'' Are workflows in modeling and
simulation the same as workflows in grid computing? Are scientific
workflows actually the same as business workflows given the right
theoretical abstraction? There are no generally accepted answers for
these questions and, indeed, every relevant reference for this work
provided a different definition of a workflow, (albeit some of the
definitions were similar).

These basic, fundamental questions have significant implications for
long-term sustainability, interoperability, and (workflow)
reproducibility. Interoperability in particular is interesting because
of the opportunity to leverage existing capabilities and workflows in
different systems without reproducing either the components or the
workflows in the second system. In practice, this is rarely possible.
Few workflow systems will work together because most systems are
developed with different or custom approaches, or, in many cases, were
never developed to be interoperable at all and there are no obvious ways
to ``just make it work.''

In this work, we propose that the solution to these problems stems from
the lack of a comprehensive ontology for workflow science. Specifically,
we

\begin{itemize}
\itemsep1pt\parskip0pt\parsep0pt
\item
  provide a brief literature review of workflows and efforts to develop
  taxonomies (not ontologies) for workflows and workflow management
  systems, and describe why a classic taxonomy may be insufficient.
\item
  present the Eclipse Integrated Computational Environment as an
  appropriate test bed for these ideas and contextualize it in the
  broader workflows tool space.
\item
  describe challenges the workflows community is currently facing that
  would be solved by the development of an ontology that facilitated
  interoperability.
\item
  enummerate the necessary tasks and challenges of such an effort.
\end{itemize}